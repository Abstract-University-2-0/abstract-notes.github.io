\input{../../../preamble.tex}
\parindent 0px

% \DeclareMathOperator{\lrhimani}{\underset{\Pi}{\underline{\int}}}
% \DeclareMathOperator{\urhimani}{\underset{\Pi}{\overline{\int}}}
% \DeclareMathOperator{\rhimani}{\underset{\Pi}{\int}}

% \DeclareMathOperator{\Kerr}{Ker}
% \DeclareMathOperator{\Imm}{Im}
% \DeclareMathOperator{\Int}{Int}
% \DeclareMathOperator{\Mat}{Mat}
% \DeclareMathOperator{\rank}{rank}
% \DeclareMathOperator{\diam}{diam}
% \DeclareMathOperator*{\id}{id}

% \newcommand{\R}{\mathbb{R}}
% \renewcommand{\C}{\mathbb{C}}
% \newcommand{\Q}{\mathbb{Q}}
% \newcommand{\N}{\mathbb{N}}
\usepackage{amsfonts, amssymb, amsmath, mathtools, amsthm}  %% for math symbs
\usepackage{mathrsfs}

\begin{document}
чё

$G \subset \R^n $ откр., огр., $f: G \to R$ огр и п.в. непр.

$\text{supp} f \subset G$

$$\implies \exists \int_G f$$

\begin{theorem}
    $G \subset \R^n$ откр., огр., 
    
    $g: G \to \R^n$ диффеоморфизм
    
    $g(G)$ огр., 

    $f: g(G) \to \R$ огр. и п.в. непр., $\text{supp}f \subset g(G)$

    Тогда $$\exists \int_G f\circ g | \det g' | = \int_{g(G)}f$$
\end{theorem}

\begin{proof}
    \begin{lemma}
        $G \subset \R^n$ открыто, $g: G \to R^n$ гомеоморфизм (биекция, непр. в обе стороны)
    
        $E \subset G: \overline{E} \subset G$

        Тогда 
        
        $\begin{matrix*}[l]
            g(\overline{E}) = \overline{g(E)} \\
    
            g(Int) = Int g(E)\\
    
            g(G\setminus\overline{E}) = g(G) \setminus \overline{g(E)}\\
    
            g(\delta E) = \delta g(E) 
        \end{matrix*}$

        Если $G, g(G)$ огр., $g$ - диффеоморфизм, то $\mu(E) = 0 \iff \mu(g(E))=0$
    \end{lemma}    
    \begin{proof}
        $G \ni x_n \to x \in G, n \to \infty \iff 
        g(G) \ni g(x_n) \to g(X) \in G(G), n \to \infty
        $

        $g(x) \ni g(\overline{E}) \iff 
        x \in \overline{E} 
        \iff g(x) \in \overline{g(E)}$

        $x \in Int E \iff \exists \varepsilon > 0 : B_\varepsilon(x) \subset E
        \iff \exists \delta > 0 : B_0 (g(x)) \subset g(E) 
        \iff g(x) \in Int g(E)$

        это $g, g^{-1}$ непр.

        $x \in \delta E \iff E \ni y_n \to x, n \to \infty $

        $G \setminus E \ni z_n \to x, n\to \infty$

        $\iff
        \begin{cases}
            g(E) \ni g(y_n) \to g(x), n \to \infty \\
            g(G\setminus E) \ni g(z_n) \to g(x)
        \end{cases}
        \iff g(x) \in \delta g(E)
        $

        $x \in G \setminus\overline{E} = Int(G \setminus E) \iff 
        g(x) \in g(x) \in Int(g(G)\setminus g(E)) = g(G \setminus \overline{g(E)})$
    \end{proof}

    $g$ - диффеоморфизм, $G, g(G)$ огр., $\mu(E)=0$

    $\forall \varepsilon \exists C_l, l=1,\dotsc, N $ открытые кубы $E \subset 
    \bigcup^\infty_{l=1}C_l, \sum^\infty_{l=1}v(C_l) < \varepsilon$

    $l(C)$ - длина ребра куба $C$, $v(C) = (l(C))^n$

    $\text{diam} C = l(C) \cdot \sqrt{n}$

    Если $\varepsilon < (\frac{\delta}{2\sqrt{n}})^n $
    
    $\forall l \quad v(C_l) < \varepsilon \implies l(C_l) < \frac{\delta}{2\sqrt{n}}$

    $\text{diam} C_l < \frac{\delta}{2}$

    $\text{dist} (E, \delta G) = \delta$

    $E \subset \bigcup^\infty_{l=1}C_l \subset G$

    $\bigcup^\infty_{l=1} C_l \subset E^{\frac{\delta}{2}} = 
    \{  x \in \R^n \mid \text{dist}(x,E) < \frac{\delta}{2} \} \subset 
    \overline{E^{\delta /2}} \subset G$

    $\max_{\overline{E^{\delta / 2}}} ||g'|| = M$

    $\forall x_1, x_2 \in E^{\delta / 2} \quad || g(x_1) - g(x_2) || < M \cdot ||x_1 - x_2||$
    
    \quad $x_l$ - центр куба $l$
    
    $G(C_l) \subset B_{M \text{diam}C_l \cdot 0.5} (g(x_l)) \subset \widetilde{C_l},
    l(\widetilde{C_l}) = M \cdot \text{diam}C_l$

    $v(\widetilde{C_l}) = (l(\widetilde{C_l}))^n = M^n (\text{diam}C_l)^n = 
    M^n (\frac{\text{diam}C_l}{\sqrt{n}})^n (\sqrt{n})^n = (M\sqrt{n})^ v(C_l)
    $

    $g(E) \subset \bigcup^\infty_{l=1}\widetilde{C_l}, 
    \sum^\infty_{l=1}\widetilde{C_l} = (M\sqrt{n})^n\cdot \varepsilon$ 

    \begin{remark}
        В условии теоремы $\exists \int_G f \circ g | \det g' |$
    \end{remark}
    \begin{proof}
        $\text{supp} f = \{ y \in g(G) \mid f(y) \neq 0\}$

        $\text{supp} (f\circ g \underbrace{| \det g' |}_{>0}) = \text{supp}f\circ g = 
        \{ x \in G \mid (f \circ g)(x) \neq 0 \}
        $

        $g(\{ x \in G | f \circ g \neq 0 \}) = \{ y \in g(G) | f(y) \neq 0\}$

        $\implies $ л.12 замыкания совпадают

        $\text{supp} (f \circ g |\det g'|)$ компактен

        $\sup_{\text{supp}(f\circ g |\det g'|)} |\det g'| < \infty \implies 
        f \circ g |\det g'|$ огр.

        \par $ $

        $f$ п.в. непр. на $g(G) \implies f \circ g $ п.в. непр. на $G$

        $| \det g'| \in C(G) \implies f \circ g |\det g'| $ п.в. непр. на $G$

        Значит, $\exists_G f\circ g |\det g' |$
    \end{proof}
    
    Разобъем область на маленькие кусочки, на каждом кусочке локально сможем представтиь
    диффеоморфизм в виде композиции простейших диффеоморфизмов, меняющих одну коорд.
    \begin{lemma} % 13 
        $G \subset \R^n$ откр., 

        $g: G \to R$ диффеоморфизм

        $\forall x \in G \quad \exists$ окрестность $U \subset G$:
        $$g | \underset{ U}{ } = g_1 \circ \dotsc \circ g_n$$
        $g_k$ - диффеоморфизм простейшего вида, т.е. 
        $(g_k)_i (x) = x_i, \forall i \neq k$
    \end{lemma}

    \begin{proof}
        Индукция 

        база: $k=1:$ уже простейший

        переход: $(g(x))_i = x_i, \quad i \ge k+1$

        $x = (y = (x_1, \dotsc, x_k),z)$

        Пусть $x_0 \in G$, $g'(x_0) = 
        \begin{pmatrix}
            \frac{\partial g}{\partial y} & \frac{\partial g}{\partial z} \\
            0 & I_{n - k}   
        \end{pmatrix}
        $

        $0 \neq \det g'(x_0) = \det \frac{\partial g}{\partial y}(x_0) \implies $
        не все миноры порядка $k-1$ нулевые. Найдётся $k-1$ независимый столбик.
        
        Перенумеровкой компонент тобъемся того, чтобы главный минор не был равен 0
    
        $f: G \to \R^n, \quad (f(x))_i = \begin{cases}
            (g(x))_i, & i < k \\
            x_i, & i \ge k
        \end{cases}$

        $f'(x_0) = \begin{pmatrix}
            \frac{\partial g}{\partial y} & \frac{\partial g}{\partial z}_{k-1} \\ 
            & I_{n - k + 1}
        \end{pmatrix}$

        $\det f'(x_0) = (\frac{\partial g}{\partial y})_{k-1} \neq 0$

        \par $ $

        $f \in C^1 (G)$

        $\exists$ окрестность $U \ni x_o \quad f|\underset{U}{ }$ - диффеоморфизм

        Рассм $h = g \circ (f|\underset{U}{ })^{-1}$ - диффеоморфизм

        $h : f(U) \to g(U)$

        $g|\underset{U}{ } = h \circ f|\underset{U}{ }$

        Для $f \, \exists$ окрестность $x_0$, в которой $f$ раскладывается в композицию
        простейших по инд предп. 

        $u \in F(U)$

        $i<k\quad (h(u))_i = (g \circ f^{-1}(u))_i = (f\circ f^{-1}(u))_i = u_i$

        $i>k\quad (h(u))_i = (g \circ f^{-1}(u))_i = (f^{-1}(u))_i = u_i$

        $(\underbrace{f(x)}_{=u})_i = x_i = (f^{-1}(u))_i \quad i>k$

        т.е. $h$ - простейший диффеоморфизм и $g$ раскладывается

    \end{proof}

    \begin{lemma} % 14 
        Утверждение теоремы верно при $n=1$

        \begin{lemma} % 14'
            В условиях теоремы на $G$ и на $g$ при $n=1$ для $\forall f: g(G)\to \R$ огр.
            : $\text{supp} \subset g(G)$

            $$\underline\int_G f\circ g |\det g'| = \underline\int_{g(G)}f 
            ,\quad
             \overline\int_G f\circ g |\det g'| = \overline\int_{g(G)} f
            $$
        \end{lemma}
        \begin{proof}
            $\text{supp}f$ компактен. 
            
            $\forall x \in \text{supp} f \quad \exists \varepsilon_x > 0 :
            [x - \varepsilon_x, x + \varepsilon_x] \subset G$

            $\text{supp}f \subset \bigcup_{x \in \text{supp}f} (x- \varepsilon_x, x + \varepsilon_x)
            \implies \text{supp}f \subset \bigcup^N_{i=1}(x_i - \varepsilon_{x_i}, x_i + \varepsilon_{x_i})
            $

            $\text{supp} f \subset \bigcup^N_{i=1}[x_i - \varepsilon_{x_i}, x_i + \varepsilon_{x_i}]
            $отрезки не пересекаются

            $\forall i \quad (g^{-1})' |_{[x_i - \varepsilon_{x_i}, x_i + \varepsilon_{x_i}]}$ имеет постоянный знака

            $g^{-1}([x_i - \varepsilon_{x_i}, x_i + \varepsilon_{x_i}])$ - отрезок

            $\implies $ достаточно доказать, что $\int_{g([a,b])}f = \int_{[a,b]} f\circ g |g'|$

            $g' > 0 : \int_{g(a)}^{g(b)}f(y)dy = \int^b_a f(g(x))g'(x)dx$

            $g' < 0 : \int_{g(b)}^{g(a)}f(y)dy = \int^b_a f(g(x)) \,\, | g'(x) | dx$

            $P_y$ - разбиение $g([a,b]) \quad P_x = g^{-1}(P_y)'' = \{ g^{-1}(\pi) | \pi \in P_y\}$

            $\max_{[a,b]} | g'| \cdot d(P_y) \le d(P_x) \le \max_{g([a,b])} | (g^{-1})'| \cdot d(P_y)$
            
            $\sum_{\pi_y \in P_y}\sup_{\pi_y}f \cdot v(\pi_y) = 
            \sum_{\pi_x \in P_x}\sup_{\pi_x} (f \circ g) |g'(\xi(\pi_x))| \cdot v(\pi_x)$

            $\pi_y = g(\pi_x)$   

            $v(\pi_y) = |g(\beta) - g(\alpha)| = |g'(\xi)| \cdot (\beta\alpha)$

            прадалжаем неравенство (не правильное он все стер суака) $\displaystyle\le \underbrace{\sup_{[a,b]}|g'|}_{<\infty} \sum_{\pi_x \in P_x}\sup_{\Pi_x}f\circ g \cdot v(\pi_x)$

            $P_{x,k}:d(P_{x,k}) \to 0, k \to \infty$

            $P_{y,k}:d(P_{y,k}) \to 0, k \to \infty$

            $\forall \pi_x \exists \xi(\pi_x) < \pi_x \quad v(\pi_y) = |g'(\xi(\pi_x))|\cdot v(\pi_x)$
            
            $U(f, P_{y,k}) \le \sum_{\pi_x \in P_x}\sup_{\pi_x}f\circ g |g'| \cdot v(\pi_x)
            = U(f\circ g |g'|, P_{x,k})
            $

            слева $\overline\int_{g([a,b])}f \le$ справа $\int_{[a,b]}f\circ g |\det g' |$

            $\overline\int_{g^{-1}g([a,b])}f\circ g |g'| \le \overline\int_{g([a,b])}f\circ g
            \circ g^{-1} | g' \circ g^{-1}|\cdot |g^{-1}|$
            
            $\implies \overline\int_{g([a,b])}f = \overline\int_{[a,b]}f\circ g |g'|$
        \end{proof}
    \end{lemma}

    \par $ $

    1. $g$ - простейший диффеоморфизм
    $\quad (g(x))_i = x_i \quad i<n$

    $g(G) = \Pi_y = \underbrace{\Pi_1}_{\subset \R^n} \times \underbrace{\Pi_{yn}}_{\subset \R}$

    $G = \Pi_x = \Pi_1 \times \underbrace{\Pi_{xn}}_{\subset \R}$

    $\int_{g(G)}f = \int_{\Pi_y} f \cdot \chi_g(G) \overset{\text{фубини}}{=} \int_{\Pi_1} dy_1 \dotsc
    dy_{n-1}\underline\int_{\Pi_2} dy_n f\cdot \chi_g(G) = \\
    \int_{\Pi_1}dy_1 \dotsc dy_{n-1}\underline\int_{
        \underbrace{\Pi_n[g(G)\cap(y_1, \dotsc,y_{n-1})\times \R]}_{\text{открытое огр мнжво}}
        }f(y)dy_n$

    $\overset{\text{лемма 14}}{=} \int_{\Pi_1}dx_1 \dotsc dx_{n-1}\underline\int_{
        \Pi_n[G \cap (y_1, \dotsc, y_n) \times \R]
    } (f\circ g)(x)(\frac{\partial g_n}{\partial x_n}(x)) dx_n$
    что-та в последней скобке равно $\det g'(x)$

    $= \int_{\Pi_1}dx_1 \dotsc dx_{n-1}\underline\int_{\Pi_{x_n}}f\circ g |\det g'| \cdot \chi_G = 
    \int_{\Pi_x} f\circ g |\det g'| \cdot \chi_G = - \int_G f \circ g |\det g'|
    $

    \par $ $
    видимо некст раз докажем

\end{proof}

\end{document}