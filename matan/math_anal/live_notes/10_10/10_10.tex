\input{../../../preamble.tex}
\parindent 0px

% \DeclareMathOperator{\lrhimani}{\underset{\Pi}{\underline{\int}}}
% \DeclareMathOperator{\urhimani}{\underset{\Pi}{\overline{\int}}}
% \DeclareMathOperator{\rhimani}{\underset{\Pi}{\int}}

% \DeclareMathOperator{\Kerr}{Ker}
% \DeclareMathOperator{\Imm}{Im}
% \DeclareMathOperator{\Int}{Int}
% \DeclareMathOperator{\Mat}{Mat}
% \DeclareMathOperator{\rank}{rank}
% \DeclareMathOperator{\diam}{diam}
% \DeclareMathOperator*{\id}{id}

% \newcommand{\R}{\mathbb{R}}
% \renewcommand{\C}{\mathbb{C}}
% \newcommand{\Q}{\mathbb{Q}}
% \newcommand{\N}{\mathbb{N}}
\usepackage{amsfonts, amssymb, amsmath, mathtools, amsthm}  %% for math symbs
\usepackage{mathrsfs}

\begin{document}
    Мы закончили на теореме Фубини

    \par $ $

    \textbf{Зам.}
    $\Pi = \Pi_1 \times \Pi_2 \subset \R^n \times \R^m, f: \Pi \to \R$ ограничена и
    непрерывна (интегрируема)
    
    \par $ $
    
    1.  $\displaystyle\int f = \int_{\Pi_2} dy \underline{\int_{\Pi_1}} f(x,y)dx = 
            \int_{\Pi_2} dy \overline{\int_{\Pi_1}} f(x,y )dx 
        $ \\
    2.  $\displaystyle\forall y \in \Pi_2 \exists \int_{\Pi_1} f(x,y)dx \implies
            \int_\Pi f = \int_{\Pi_2}dy\int_{\Pi_1}f(x,y)dx
        $

    \par $ $

    \textbf{Пример.}

    $\Pi_1 = \Pi_2 = [0,1] \quad f(x,y) = 
        \begin{cases}
            1, x\in [0, 1]\setminus \Q, 
            \text{ или } y \in [0,1] \setminus \Q \\
            1 - \frac{1}{q}, x = \frac{p}{q}, y \in [0, 1] \cap \Q 
        \end{cases}
    $

    \par $ $

    $f$ непрерывна на $([0, 1]\setminus \Q)^2$. Это счетное множество, короче
    там чета с мерой нуль

    $(x,y) \in ([0, 1] \setminus \Q)^2, \quad f(x,y) = 1$ \\
    $\epsilon, q: \frac{1}{q} < \epsilon$. 

    Отметим рациональные числа со всеми знаменателями от 1 до $q$. 
    это будет какая-то решётка точек с каким-то наименьшим расстоянием между точками. 
    $x$ не попадёт, потому что он иррационален, к нему будет какое-то ближайшее число,
    то есть найдется окрестность икса, что туда попадут чета. найдём окрестность точки
    (x,y) 
    
    $\forall \epsilon \, \exists Q: \frac{1}{Q} < \epsilon \land \forall q > Q: |1 - \frac{1}{q} - 1| < \epsilon$
    
    $f $ почти везде непрерывна на $[0,1]^2 = \Pi$, огр. $\implies \exists \int_\Pi f$

    $x \in [0, 1] \setminus \Q: \quad \underline{\int_{\Pi_2}} f(x,y)dy = 
        \overline{\int_{\Pi_2}} f(x,y) dy = \int_{\Pi_2} 1 = 1 \\
        x \in [0, 1] \cap \Q: \underline{\int_{\Pi_2}}f(x,y)dy = 1 - \frac{1}{q}, \quad
        \overline{\int_{\Pi_2}} f(x,y) dy = 1
    $

    \par $ $
    
    $\mathscr{L}(x) = \begin{cases}
        1, x \in [0, 1]\setminus \Q \\ 
        1 - \frac{1}{}q, x = \frac{p}{q}, \text{ непр}\\
    \end{cases}
    $

    $\mathscr{U}(x) = 1$
    
    \par $ $

    % Т. Фубини утв.,
    $\int_{\Pi_1} \mathscr{L} = 1 = \int_{\Pi_1} U - \int_\Pi f$
    
    \par $ $

    $E \subset \Pi = [a, b] \times [c, d], \mu(\delta E) = 0 \\
    f \in C(E), \quad \widetilde{f} = f \cdot \chi_E \\
    \int_E f = \int_\Pi \widetilde{f} = \int^b_a dx \int^d_c \widetilde{f}(x,y)dy =
    \int^d_cdy\int^b_af(x,y) dx 
    \\\\
    E = \{ (x,y) \in \Pi | a \le x \le b, y_1 (x) \le y \le y_2(x) \} = \\
    \{ (x,y ) \in \Pi | c \le y \le d, x_1(y) \le x \le x_2(y)\}
    \\\\
    \dotsc \int_E f = \int^b_a dx \int^{y_2(x)}_{y_1(x)} f(x,y)dy = 
    \int^d_cdy\int^{x_2(y)}_{x_1(y)}f(x,y)dx
    $
    
    \begin{remark}
        $f \in C([0, 1])$ 

        \par $ $ 

        Тогда $\mu (\underbrace{\text{график }}_{ \{ (x, f(x)) | x \in [0, 1]\} } f) = 0$
    \end{remark}
    \begin{proof}
        $[0, 1]$ компакт $\implies f$ равномерно непр. на $[0, 1]$, $\epsilon > 0 \,\exists \delta > 0:
        \forall x_1, x_2 \in [0, 1]: |x_1 - x_2| < \delta \quad |f(x_1) - f(x_2)| < \epsilon
        $

        $[\frac{2}{\delta}] + 1$ интервалов (интервалы длины $\frac{\delta}{2}$)
        
        $([\frac{2}{\delta}] + 1) \cdot 2\epsilon \cdot \frac{\delta}{2} < 
        4\epsilon \frac{\delta}{2} [\frac{2}{\delta}] < 4\epsilon
        $ -- площадь что-ли

        $\forall \epsilon \exists$ покрытие квадратами, сумма площадей которых
        не превосходит ээээ двух чё-то там ээээ $\sum v(c) < 8\epsilon$
    \end{proof}

    % \rule{0.2px}{\textwidth} pizd 
    Когда мы говорили про интегрируемость по множеству мы определили интеграл 
    по Е почти везде непрерывный... измеримость по жордану что-то...

    $\mu(E) = 0, f: E \to \R$ почти везде непрерывна, ограничена 
    $\cancel\implies \int_E f = 0$
    \par $ $

    $E = [0, 1] \cap \Q, f \equiv 1, f: E \to \R $

    $$\widetilde{f}(x) = f(x) \cdot\chi_E(x) = \begin{cases}
        1, x \in E \\
        0, x \in [0, 1] \setminus \Q
    \end{cases}
    $$

    $$\cancel\exists \int_{[0, 1]}\widetilde{f}, \exists \int_E f$$

    $v(E) = 0 \implies E$ измеримо по жордану и его жорданов объём равен 0
    \begin{proof}
        $\displaystyle v(E) = 0. \quad\forall \epsilon > 0 \, \exists C_k, k=1,\dotsc,N \text{ замыкание (кубы)}:
        E \subset \bigcup^N_{k=1} C_k, \sum^N_{k=1} v(C_k) < \epsilon
        $

        \par $ $

        $\delta E \subset \overline{E} \subset \bigcup^N_{k=1} C_k$

        $\implies v(E) = 0, v(\delta E) = 0 \implies \mu(\delta E) = 0 \implies E$ 
        измеримо по жоржану
        
        $\exists \Pi, E \subset \Pi \quad \forall \epsilon E \subset \bigcup^N_{k=1}C_k$

        Можно считать, что $\forall k \, C_k \subset \Pi$. Пусть $P$ - разбиение $\Pi$ гранями
        всех $C_k$

        Оценим интеграл 
        
        $\displaystyle v(E) = \int_E 1 = \int_\Pi \chi_E \le 
        U(\chi_E, P) \, \forall P = 
        \sum_{\Pi \in P} \sup_\Pi \chi_E \cdot v(\Pi)
        = \\\sum_{\underset{\Pi \in \bigcup^N_{k=1}C_k}{\Pi \in P}} v(\Pi) \le 
        \sum^N_{k=1} v(C_k) < \epsilon \quad \forall \epsilon\\\\
        \implies v(E) = 0
        $
    \end{proof}

    \begin{lemma} % 10
        $\Pi \subset \R^n, f_1 ,f_2: \Pi \to \R$ огр., п.в. непрерывна
        $\implies a_1f_1 + a_2f_2$ -- огр., п.в. непрерывна

        $$\int_\Pi(a_1f_1 + a_2f_2) = a_1\int_{\Pi_1} f + a_2\int_{\Pi_2}f$$
    \end{lemma}

    \begin{proof}
        $\measuredangle P, \Xi$ 

        $\sum(a_1f_1 + a_2f_2,P , \Xi) = \sum(a_1f_1(\xi (\Pi))  + a_2f_2(\xi(\Pi)) )\cdot v(\Pi) =
        a_1 \sum(f_1, P, \Xi) + a_2 \sum (f_2, P, \Xi)
        $ ёбббб хзх фотка 

        Пусть $P_k, k \in \N, d(P_k) \to 0, k \to \infty, \Xi_k, k \in \N: 
        \int_\pi(a_1f_1 + a_2f_2) = a_1\int_\Pi f_1 + a_2\int_\Pi f_2
        $
    \end{proof}

    \begin{lemma} % 11
        $E_1, E_2$ измеримы по жордану и не пересекаются

        $f: E_1 \cup E_2 \to \R$ огр и п.в. непр.

        $$\int_{E_1 \cup E_2} f = \int_{E_1} f + \int_{E_2} f$$
    \end{lemma}
    \begin{proof}
        $\widetilde{f} = f \cdot \chi_E, \, \Pi \supset E_1 \cup E_2$ 
        \par $ $

        $\displaystyle \int_{E_1 \cup E_2} f = 
        \int_\Pi f \cdot \chi_{E_1 \cup E_2} = 
        \int_\Pi (f \cdot \chi_{E_1} + f \cdot \chi_{E_2}) = 
        \int_\Pi f \cdot \chi_{E_1} + \int_\Pi f \cdot \chi_{E_2} =
        \int_{E_1} f + \int_{E_2} f
        $
    \end{proof}
    \par $ $

    \begin{lemma} % 12
        $\Pi \subset \R^n, f: \Pi \to \R$ огр и п.в. непр.

        $$\left|\int_\Pi f \right| \le \int_\Pi | f |$$
    \end{lemma}
    \begin{proof}
        $P_k, d(P_k) \to 0, \Xi_k$
        $$
        \underbrace{\left| \sum(f, P_k, \Xi) \right|}_{\xrightarrow[k \to \infty]{} |\int_\Pi f|} \le  
        \left| \sum_{\Pi \in P_k} f(\xi(\Pi)) \cdot v(\Pi) \right| \le 
        \underbrace{\sum_{\Pi \in P_k} \left| f(\xi(\Pi)) \right| \cdot v(\Pi) }_{
            \xrightarrow[k \to \infty]{} |\int_\Pi f|
        }
        $$
    \end{proof}
    \begin{lemma} % 13
        $v(E) = 0, f: E \to \R$ огр 

        $\implies \int_E f  = 0$
    \end{lemma}
    \begin{proof}
        $\displaystyle\forall \epsilon \, \exists \bigcup^N_{k=1}C_k, 
        \sum^N_{k=1}v(C_k) < \epsilon$

        $\exists M > 0: \forall x \in E \quad |f(x)| < M \implies 
        | \widetilde{f}(x) | < M, \forall x \in \Pi$

        $C_k, k=1, \dotsc, N \to$  разбиение $P$

        $\displaystyle
        | U(f, P) | = | \sum_{\pi \in P} \sup_\Pi |f|\chi_E \cdot v(\pi) \le 
        \sum_{\underset{\Pi \in \bigcup^N_{k=1} C_k}{\Pi \in P}} \sup |f|\cdot v(\Pi) \le
        M \cdot \sum^N_{k=1}v(C_k) \le M\epsilon$

        $|\int_E f| = \int_\Pi f\cdot\chi_E \le 
        \int_\Pi |f|\cdot \chi_E \le 
        U(|f|, \chi_E, P) = \sup ... $
        \par $ $

        $\epsilon$ произволен $\implies \int_E f = 0$
        
    \end{proof}

    \textbf{Замена переменной в интеграле}
    
    $E \subset \R^n \quad f: E \to \R$
        
    носитель $\text{supp} f = \overline{\{ x: f(x) \neq 0\}}$ (замыкание)
    \small{//носитель компактен}

    \begin{remark}
        Пусть $G \subset \R^n$ открыто и ограничено

        $f: G \to \R$ огр. и п.в. непр.

        Если $\text{supp} f \subset G$ то $\exists \int_G f$ 
        (независимо от $\delta G$)
    \end{remark}
    \begin{proof}
        $\exists \Pi: G \subset \Pi, supp f \subset Int \Pi, \widetilde{f}: \Pi \to \R$
        продолжение $f$ нулём на $\Pi$

        $\underbrace{\{ \text{т. разрыва } \widetilde{f}\}}_{
            \subset \{ \text{т.р. } f \text{ на } G\} - \text{ мера } 0
            } = \{ \text{т.р. } \widetilde{f} \text{ на } supp f\}
        \cup \{ \text{т.р. } \widetilde{f} \text{ в } \underbrace{Int \Pi\setminus supp f\}}_{\text{откр.}}
        \cup \underbrace{\{ \text{т.р. } \widetilde{f} \text{ на } \delta\Pi\}}_{= \varnothing}
        $
        \par $ $
        
        $dist\{ \delta \Pi, supp f\} > 0 \\
        \widetilde{f} \mid_{\Pi \setminus supp f} \equiv 0
        $

        Второе множество в объединении тоже пусто $\widetilde{f} \equiv 0$
    \end{proof}

    \begin{theorem}
        Пусть $G\subset \R^n $ открыто и ограничено, $g: G\to\R^n$ диффеоморфизм,

        $g(G)$ ограничен

        $f: G \to \R$ ограничена и п.в. непрерывна

        $supp f \subset g(G)$

        Тогда $\exists \int_G f \circ g | \det g' |$ и $\int_{g(G)} f = 
        \int_G f\circ g \cdot |\det g'|$
    \end{theorem}
    \textbf{Опр.}
    $G$ называется областью, если оно открыто и связно.
    
    \begin{proof}
        
    \end{proof}
    
\end{document} 