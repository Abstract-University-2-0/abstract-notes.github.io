\input{../../preamble.tex}
\parindent 0px

\begin{document}
    \textbf{1.} Каноническая форма дифференциального уравнения $n$-го порядка 
    (системы дифференциальных уравнений первого порядка). 

    Постановка задачи Коши для дифференциального уравнения n-го 
    порядка и для системы дифференциальных уравнений
    первого порядка, дуализм этих задач.
    
    \rule{\textwidth}{1px}
    
    \par $ $
    $$F(x, y(x), y'(x), \dotsc, y^{(n)}(x)) = 0 \quad\textbf{ -- общий вид}$$
    $\displaystyle\frac{\partial F}{\partial y^{(n)}} \neq 0$, 
    $n$ -- порядок дифференциального уравнения. 
    
    \,\\
    По теореме о неявной функции: $y^{(n)} = F(x, y, \dotsc, y^{(n-1)})$
    \textbf{-- канонический вид}

    % $y^{(n)} = f(x, y, \dotsc, y^{(n-1)})$

    \,\\
    $
    \begin{cases}
        y(x_0) = y_0 \\
        y'(x_0) = y_1 \\
        \dotsc \\
        y^{(n-1)(x_0) = y_{n-1}}
    \end{cases}
    \iff
    \begin{cases}
        u'_1=u_2 \\
        u_2'=u_3 \\
        \dotsc \\
        u'_{n-1} = u_n \\
        u'_n = F(x, \vec{u})
    \end{cases}
    \overset{\overset{\text{замена переменных}}{\downarrow}}{u(x)}= 
    \begin{cases}
        y(x) \\
        y'(x) \\
        \vdots \\
        y^{(n-1)}(x)
    \end{cases}
    $

    \,\\
    Каждое решение уравнения 1 системы переходит с помощью замены 
    в решение второй системы.

    Задача Коши.
    $\sqsupset y^{(n)} = F(x, y, \dotsc, y^{(n-1)}) \qquad 
    \begin{matrix*}[l]
        t \in D \subset \mathbb{R} \\ y_0, \dotsc, y_{n-1} \in \mathbb{R}
    \end{matrix*}$
    
    Тогда 
    $\begin{cases}
        y(x_0) = y_0 \\
        \dot{y}(x_0) = y_1 \\
        \dotsc \\
        y^{(n-1)}(x_0) = y_{n-1}
    \end{cases}$ имеет решение в $\varepsilon$-окрестности точки $x_0$
\end{document}